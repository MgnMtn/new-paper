% The title of the paper
\newcommand{\papertitle}{A New Paper}

% The authors of the paper.
% Usually different conference formats have different 
% ways of including this information, so this command
% is not needed.
\newcommand{\paperauthors}{
    Phil McMinn
}

\documentclass[12pt]{article}
\newcommand{\paperbibliographystyle}{plain}


% Packages used by the paper

\usepackage{booktabs}
\usepackage{graphicx}
\usepackage{xspace}
% Custom commands used by the paper

% Suppress details if double-blind, else write them out
\newcommand{\ifnotdoubleblind}[1] {
    \ifcsname doubleblind\endcsname
        \doubleblind
    \else 
      #1 
    \fi
}
% Constants that appear throughout the paper

\newcommand{\numsubjects}{}
% Words and phrases that need to be spelt
% and formatted consistently

\newcommand{\etc}{etc.\xspace}
\newcommand{\etal}{et~al.\xspace}
\newcommand{\ie}{i.e.\xspace}

\begin{document}

% The \papertitle command lives in details.tex
% It is useful to have it defined as a command so that
% it can be referenced in other documents,
% e.g. a referee response letter
\title{\papertitle}

% The \paperauthors command lives in details.tex
% A publisher's template may have different
% commands for adding the authors to a paper
\author{\paperauthors}

% This is usually not needed:
\date{}

\maketitle

% Sections of the document. 
% These are some typical ones.
% Add more if you need them.
\begin{abstract}
    
The file structure that produce this PDF are intended to be a starting point for
your own paper.
%
It contains examples of how to structure figures and tables, and tips as to what
to write in each section of the paper.
%
In the abstract, you should, as succinctly as possible, introduce the problem
and your approach, and give an overview of the results of your empirical study.
%
End with some figures that summarise your results, and which give hard evidence
of the benefits of your technique.
%
Note that some conferences and journals limit the number of words in an abstract
to 200 or 250.

\end{abstract}

\section{Introduction}
\label{sec:introduction}

This section sets the stage for the paper. 
It should clearly explain the problem addressed by the 
paper, and give high level details of your approach
for tackling it, explaining why no other paper already
addresses the exact same problem.

Right before this section concludes, it should always
feature a list of contributions that the paper makes:

\begin{enumerate}

    \item Contribution 1. Description.

    \item Contribution 2. Description.

\end{enumerate}    

Try to avoid concluding the introduction with a 
``This paper is organized as follows\ldots'' paragraph,
as they are rarely insightful! 
%
However, you may want to include a sentence or two 
leading into the next section --- Background.
\section{Background}
\label{sec:background}

The background section introduces past work (by you
and others) that the reader needs to know to understand 
the rest of your paper. 
%
It is different from the ``Related Work'' section, 
% Always reference sections like this, i.e. with
% a capital letter on "Section", followed by a 
% non-breaking space (the ~ character), which
% keeps the word "Section" and the following
% number on teh same line.
Section~\ref{sec:related-work},
in which you should cite work related to your paper,
but which is not integral to the basis of your approach.

% You can copy this file for every new figure.
%
% Figures should be placed at the top of pages/columns
% where they can.
%
% This can be ensured by using the [t] parameter to the
% "\begin{figure}" declaration.
%
\begin{figure}[t]
    % Figures should be centered in the page/column
    \centering

    % Figure content goes here.
    A figure.

    \caption{
        % The label should appear _inside_ the caption to ensure
        % Latex numbers it correctly. This is a common gotcha!
        %
        % All figure labels should start with "fig:"
        % So that the figure file can be found easily, the rest of the
        % figure label should be the same as the filename, as it is
        % in this example:
        %
        \label{fig:plain-figure}
        %
        Add your caption here. Captions for figures go {\em below}
        the figure.
    }
\end{figure}



\section{Approach}
\label{sec:approach}

% Your text goes here
\section{Evaluation}
\label{sec:evaluation}

Evaluation sections should include a list of research questions near the
beginning.

\subsection{Subjects}
\label{sec:subjects}

You should describe details of the subjects that you used in your study,
summarised in a table, such as the example. The information in the file ``{\tt
figures/plain-figure.tex}'' describes how to format a figure, included here as
% Always reference tables like you would reference sections/figures, with a
% capital "T" for table a non-breaking space (the ~ character) between "Table"
% and "\ref" to ensure the word and the number are not split over two lines.
Table~\ref{tab:plain-table}.

\subsection{Methodology}
\label{sec:methodology}

This section should then include a description of the methodology used to answer
them, in a dedicated subsection, like this.

You should also include a description of the tooling to implement your
technique, and run the experiments.

\subsection{Threats to Validity}
\label{sec:threats-to-validity}

The section should end with a discussion of the threats to validity of the
experimental design.

Some authors choose to place this after the results. However if written {\em
before} the results, it comes across more clearly that you have carefully
thought everything through.

An example of a threat might be the type of statistical tests you used. Here,
you can explain why they were appropriate. For example, you may have used
non-parametric tests, such as the Mann-Whitney U-Test, because you could not
assume the distribution of your results was normal.

Another threat might be the types of subjects you used, whether the code you
wrote to perform was reliable (you should write tests to mitigate this threat),
and so on.

\section{Results}
\label{sec:results}

In this section, you should relate your results
to each of your research questions, one by one.

Any more anecdotal observations, or anything you
observed in the course of answering your research
questions that you did not deliberately set out 
to investigate should be included
in a subsection at the end of this section, called
``Discussion''.

% Tables should be placed at the top of pages/columns
% where they can.
%
% This can be ensured by using the [t] parameter to the
% "\begin{table}" declaration.
%
\begin{table}[t]
    % Figures should be centered in the page/column
    \centering
    %
    \caption{
        % The label should appear _inside_ the caption to ensure
        % Latex numbers it correctly. This is a common gotcha!
        %
        % All table labels should start with "tab:"
        % So that the figure file can be found easily, the rest of the
        % table's label should be the same as the filename, as it is
        % in this example:
        %
        \label{tab:plain-table}
        %
        Add your caption here. Captions for tables go {\em above}
        the table.
    }
    %
    % Depending on the template, some breathing space might need to
    % be added (use \smallskip \medskip etc.) here
    %
    % Or, to save space, you might want to remove space
    % (use a negative \vspace, e.g. \vspace{-1em})
    %
    %
    % Table content goes here. Use this file to specify the
    % table's column headings. The data should be automatically
    % output from a program processing the raw experimental data
    % and should be inputted from another file. This enables
    % the data to change, if for example, the experiment data
    % needs to be updated.
    %
    % Do not use vertical rules. Ensure you use \toprule, \midrule
    % and \bottomrule from the "booktabs" package effectively.
    %
    % Numbers should be right justified (use "r"),
    % text left justified (use "l").
    %
    % For example:
    %
    \begin{tabular}{lr}
        \toprule
            % use a new line for each column if needed
            {\bf Algorithm}  &
            {\bf Best Fitness}
            \\

        \midrule

        % Now input the data file
        % Note the filename should not have curly brackets, 
        % otherwise latex will generate a warning 
        % (see https://tex.stackexchange.com/questions/567985/problems-with-inputtable-tex-hline-after-2020-fall-latex-release
        % as to why)
        \input table-data/results

        \bottomrule
    \end{tabular}
    %
    % To save space, you might want to remove space here
    % (use a negative \vspace, e.g. \vspace{-1em})
\end{table}
\section{Related Work}
\label{sec:related-work}

% Your text goes here
\section{Conclusions and Future Work}
\label{sec:conclusions-and-future-work}

This section should begin by reminded the reader
about your approach and its motivation, and that
no other previous research addresses the same
problem.

% The \paperbibliographystyle command is defined
% by the paper's style file ... see the file
% referenced by the \input{styles/...}
% declaration at the top of this file
\bibliographystyle{\paperbibliographystyle}

% The bibliography should be a Git subproject
% imported into this repository
%
% \bibliography{...}

\end{document}