\section{Introduction}
\label{sec:introduction}

% Always edit your text without the use of word-wrap, like the paragraphs in
% this section, because:
%
% 1. Hard-breaks make it easier to track changes in version control.
% 2. It is easier to insert comments explaining some aspect of your writing
%    mid-sentence
% 3. You can break up the structure of your sentence across multiple lines so
%    that it is easier to edit later.

% If you are having trouble structuring or phrasing your text, try summarising
% the key point of what you want to get across in a leading comment, like this:

% This section sets the stage for the paper
This section should clearly explain the problem addressed by the paper, and give
high level details of your approach for tackling it, explaining why no other
paper already addresses the exact same problem.

Like the abstract, it should provide a brief summary of the results of the
paper, with some statistics if possible. If you can use some different
statistics to the abstract, then that helps vary things for the reader.

Right before this section concludes, it should always feature a list of {\em
contributions} that the paper makes. A contribution is a technical contribution
to the research literature. What things does your paper show that no other paper
does? Contributions are things like your approach/technique. One contribution
can be the empirical evaluation and its results. Again, you may wish to
highlight some key results here, with some numbers. In general, this part of the
introduction should be written in the following format:

The contributions of this paper, therefore, are as follows:

\begin{enumerate}

\item Contribution 1. Longer description (including the section where the reader
can find this contribution).

\item Contribution 2. Longer description (including the section where the reader
can find this contribution).

\end{enumerate}

For example, an entry in this list might take the form:
\begin{quote} \it The results of an empirical study showing the DOG algorithm
outperforms the CAT algorithm with a 30\% improved fitness score
(Section~\ref{sec:results}).
\end{quote}

Try to avoid concluding the introduction with a ``This paper is organized as
follows\ldots'' paragraph, as they are rarely insightful!
%
However, you may want to include a sentence or two leading into the next section
--- Background.
