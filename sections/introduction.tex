\section{Introduction}
\label{sec:introduction}

This section sets the stage for the paper. 
It should clearly explain the problem addressed by the 
paper, and give high level details of your approach
for tackling it, explaining why no other paper already
addresses the exact same problem.

Like the abstract, it should provide a brief summary of 
the results of the paper, with some statistics if possible. 
If you can use some different statistics to the 
abstract, then that helps vary things for the reader.

Right before this section concludes, it should always
feature a list of contributions that the paper makes.
One contribution can be the empirical evaluation and
its results. Again, you may wish to highlight some 
key results here, with some numbers. 

\begin{enumerate}

    \item Contribution 1. Description.

    \item Contribution 2. Description.

\end{enumerate}    

Try to avoid concluding the introduction with a 
``This paper is organized as follows\ldots'' paragraph,
as they are rarely insightful! 
%
However, you may want to include a sentence or two 
leading into the next section --- Background.