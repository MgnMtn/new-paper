\section{Background}
\label{sec:background}

The background section introduces past work (by you
and others) that the reader needs to know to understand
the rest of your paper.
%
It is different from the ``Related Work'' section,
% Always reference sections like this, i.e. with
% a capital letter on "Section", followed by a
% non-breaking space (the ~ character), which
% keeps the word "Section" and the following
% number on teh same line.
Section~\ref{sec:related-work},
in which you should cite work related to your paper,
but which is not integral to the basis of your approach.

Think of it as content behind the ``{\tt import}'' statements 
of a program. What is the minimal information that the reader 
needs to know to understand your technique? (What would 
correspond to the libraries that you would ``{\tt import}'' 
if it were a computer program?

Everything else can go in ``Related Work''. Don't put everything 
in this section, ``Background'' because it will make your
work sound more derivative and will delay the reader
before getting to the part you want to grandstand, 
which is your method, in the next secion, called 
``Approach''.

% You can copy this file for every new figure.
%
% Figures should be placed at the top of pages/columns
% where they can.
%
% This can be ensured by using the [t] parameter to the
% "\begin{figure}" declaration.
%
\begin{figure}[t]
    % Figures should be centered in the page/column
    \centering

    % Figure content goes here.
    A figure.

    \caption{
        % The label should appear _inside_ the caption to ensure
        % Latex numbers it correctly. This is a common gotcha!
        %
        % All figure labels should start with "fig:"
        % So that the figure file can be found easily, the rest of the
        % figure label should be the same as the filename, as it is
        % in this example:
        %
        \label{fig:plain-figure}
        %
        Add your caption here. Captions for figures go {\em below}
        the figure.
    }
\end{figure}


