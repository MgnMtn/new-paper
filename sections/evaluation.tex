\section{Evaluation}
\label{sec:evaluation}

Evaluation sections should include a list of
research questions near the beginning.


\subsection{Subjects}
\label{sec:subjects}

You should describe details of the subjects that 
you used in your study, summarised in a table,
such as the example. 
The information in the file 
``{\tt figures/plain-figure.tex}''
describes how to format a figure, included here as
% Always reference tables like you would reference
% sections/figures, with a capital "T" for table
% a non-breaking space (the ~ character) between
% "Table" and "\ref" to ensure the word and the
% number are not split over two lines.
Table~\ref{tab:plain-table}.



\subsection{Methodology}
\label{sec:methodology}

This section should then include a description of
the methodology used to answer them, in a dedicated
subsection, like this.



\subsection{Threats to Validity}
\label{sec:threats-to-validity}

The section should end with a discussion of the threats
to validity of the experimental design.

Some authors choose to place this after the results.
However if written {\em before} the results, it comes
across more clearly that you have carefully thought
everything through.

An example of a threat might be the type of 
statistical tests you used. Here, you can explain
why they were appropriate. For example, you may
have used non-parametric tests, such as the 
Mann-Whitney U-Test, because you could not assume
the distribution of your results was normal.

Another threat might be the types of subjects you
used, whether the code you wrote to perform was 
reliable (you should write tests to mitigate this
threat), and so on.
