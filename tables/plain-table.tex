% Tables should be placed at the top of pages/columns
% where they can.
%
% This can be ensured by using the [t] parameter to the
% "\begin{table}" declaration.
%
\begin{table}[t]
    % Figures should be centered in the page/column
    \centering
    %
    \caption{
        % The label should appear _inside_ the caption to ensure
        % Latex numbers it correctly. This is a common gotcha!
        %
        % All table labels should start with "tab:"
        % So that the figure file can be found easily, the rest of the
        % table's label should be the same as the filename, as it is
        % in this example:
        %
        \label{tab:plain-table}
        %
        Add your caption here. Captions for tables go {\em above}
        the table.
    }
    %
    % Depending on the template, some breathing space might need to
    % be added (use \smallskip \medskip etc.) here
    %
    % Or, to save space, you might want to remove space
    % (use a negative \vspace, e.g. \vspace{-1em})
    %
    %
    % Table content goes here. Use this file to specify the
    % table's column headings. The data should be automatically
    % output from a program processing the raw experimental data
    % and should be inputted from another file. This enables
    % the data to change, if for example, the experiment data
    % needs to be updated.
    %
    % Do not use vertical rules. Ensure you use \toprule, \midrule
    % and \bottomrule from the "booktabs" package effectively.
    %
    % Numbers should be right justified (use "r"),
    % text left justified (use "l").
    %
    % For example:
    %
    \begin{tabular}{lr}
        \toprule
            % use a new line for each column if needed
            {\bf Algorithm}  &
            {\bf Best Fitness}
            \\

        \midrule

        % The data in this file should be automatically output
% from code that processes raw results data.
DOG         & 78  \\
CAT         & 60  \\

\midrule

{\bf Mean}  & 69  \\

        \bottomrule
    \end{tabular}
    %
    % To save space, you might want to remove space here
    % (use a negative \vspace, e.g. \vspace{-1em})
\end{table}